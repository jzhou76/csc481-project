%==============================================================================%
\section{Edit Distance}
%==============================================================================%

\subsection{Algorithm}
We use Dynamic Programming to solve the Edit Distance problem.
We define a 2-dimensional array ED[6][6], and ED[i][j] represents the edit distance
between the first ith characters of Alice's input and the first jth character
of Bob's input; and Alice\_Input[i] means the ith character of Alice's input
and Bob\_Input[j] means the jth characters of Bob's input.
The reason we need make each dimension to be 6 characters
is that we need initialize the base case for the dynamic programming algorithm:
we initialize ED[i][0] = i because it takes i steps to make the two input
string the same if the second input is empty and the first input's size is i.
For the same reason we initialize ED[0][j] = j.

When computing ED[i][j] where neither i nor j is zero, if
Alice\_Input[i] == Bob\_Input[j], ED[i][j] = ED[i - 1][j - 1];
if not, there are three options to apply:
\begin{enumerate}
    \item Directly replacing Alice\_Input[i] with Bob\_Input[j];
        ED[i][j] = 1 + ED[i - 1][j - 1].
    \item Deleting the Alice\_Input[i]; ED[i][j] = 1 + ED[i - 1][j].
    \item Appending Bob\_Input[j] to the end of Alice\_Input[i];
        ED[i][j] = 1 + ED[i][j - 1].
\end{enumerate}

Apparently ED[i][j] should be the minimum result of the three cases.
Algorithm~\ref{alg:ed_dp} shows the process described above.

\begin{algorithm}[h]
    \begin{algorithmic}[1]
        \begin{footnotesize}
        \State ED[6][6]: E[i][j] means the edit distance between the first ith
        characters of Alice's input and the first jth character of Bob's input.
        \Function{Edit\_Distance}{Input\_String}
            % \State
            \For{Each Element E of ED}
                \Comment{Initialization}
                \State Initialize E to Integer(32, 0, PUBLIC)
            \EndFor

            \State

            \For{$i \in \{1, 2, 3, 4, 5\}$}
            \State ED[i][0] = Integer(32, i, PUBLIC)
            \EndFor
            \For{$j \in \{1, 2, 3, 4, 5\}$}
            \State ED[0][j] = Integer(32, j, PUBLIC)
            \EndFor

            \State

            \For{$i \in \{1, 2, 3, 4, 5\}$}
                \For{$j \in \{1, 2, 3, 4, 5\}$}
                    \If{Alice\_Input[i] == Bob\_Input[j]}
                        \State ED[i][j] = ED[i - 1][j - 1]
                    \Else
                        \State ED[i][j] = MIN(ED[i - 1][j], MIN(ED[i][j -
                        1],ED[j - 1][j - 1])) + Integer(32, 1, PUBLIC)
                    \EndIf
                \EndFor
            \EndFor
        \EndFunction
        \end{footnotesize}
    \end{algorithmic}
    \caption{Dynamic Programming Algorithm for Edit Distance}
    \label{alg:ed_dp}
\end{algorithm}

\subsection{Implementation}
The {\tt emp-toolkit} library does not implement a secure version of the
{\tt string} type; so to realize comparison character by character, first we
convert the input string to a sequence of the {\tt Integer} objects defined
in the {\tt emp-toolkit} library. Since an input string consists only of
at most 4 characters `A', `T', `G',`C', we map them to be number 1, 2, 3, 4,
and encode them into {\tt Integer} objects.
In this way, we can compare the characters from Alice and Bob by comparing
{\tt Integer} objects. Our implementation ensure security because after the
initial ``character to Integer'' conversion, all
the operations and their operators' types that are used to do the
Dynamic Programming are defined by the {\tt emp-toolkit} library, which means
they all can be synthesized to secure garbled circuits.
