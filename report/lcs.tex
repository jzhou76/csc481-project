%==============================================================================%
\section{Longest Common Subsequence}
%==============================================================================%

\subsection{Introduction}
The longest common subsequence(LCS) problem is to find the longest subsequence that is common in all given inputs.
It is widely used in both computer programs (e.g., file comparison program diff) and other scientific fields(e.g., linguistics).
Traditionally, in order to compute LCS, inputs are public to all parties.
However, adversities may learn additional information that individual may not want to disclose.
For example, if someone wants to know the common availability with his/her colleague, he/she needs to share the date of all available.
But the colleague could infer more information from the date, e.g., personal travel plan.
Therefore, there is a need to implement a secure version to tackle LCS problem.


\subsection{Algorithms and Implementation}
In this project, we construct the problem as follows: Alice and Bob are two parties and each of them generates a fixed length of ASCII character array.
The goal is to calculate the LCS length of two arrays without reveling the content of input to the other party.

The LCS problem is solved by Dynamic Programming.
We create a two-dimensional table({\tt lcs}) toAlice\_Input.length memoize the result.
{\tt lcs[i][j]} stores the length of lcs for two inputs' prefix {\tt Alice\_Input[0:i-1]} and {\tt Bob\_Input[0:j-1]}.
The final result is obtained from {\tt lcs[-1][-1]}.


\begin{verbatim}
for (int i = 1; i <= Alice_Input.length; i++) {
    for (int j = 0; j < Bob_Input.length; j++) {
        if (Alice_Input[i - 1] == Bob_Input[j - 1])
            lcs[i][j] = lcs[i - 1][j - 1] + Integer(32, 1, PUBLIC);
        else
            lcs[i][j] = max(lcs[i - 1][j], m[i][j - 1]);
    }
}
\end{verbatim}