%==============================================================================%
\section{Primeness}
%==============================================================================%
Prime numbers are extremely important in Cryptography. We define a new
property of a number: primeness. The primeness of a number is the number of
its prime factors. For example, $12345 = 3 \times 5 \times 823$;
so its primeness is 3.  We count the same prime factor accumulatively,
e.g., 8's primeness is 3.
We say that A is a better number than B in terms of primeness if A's primeness
is less than B's.

We define a secure 2-party computation problem: both Alice and Bob input
a number, the output is whose input is a better number in terms of primeness.
Since there is no existing efficient algorithm to factorize a big number,
we restrict the input to be less than 1,000,000.

\subsection{Algorithm and Implementation}
To make the implementation easier, we chose a simple and straightforward
algorithm to compute a number's primeness. Because the input number is
less than 1 million, we know that it has at most one prime factor that is
greater than 1,000. There are only 168 prime numbers less than 1,000.
We can iterate over these 168 numbers and use modulo operation to find out
how many a certain prime number appears in the input's factorization.
We wrote a {\tt Python} script to generate these 168 prime numbers and
put them in an array in our {\tt cpp} solution.

To compute how many times a prime number is in the factorization of the input,
intuitively we can use a simple {\tt while} loop to solve it as follows

\begin{verbatim}
while (num % prime == 0) {
    prime_count++;
    num = num / prime;
}
\end{verbatim}

However, in order to successfully build garbled circuits, the number of
iteration of a loop must be fixed during compile time. With this restriction,
we need convert the loop above with a fixed iteration number.
Luckily, for the result to be less than 1 million, the maximum exponent of
a prime is a small number. For example, $2^{19} < 1,000,000$ and
$2^{20} > 1,000,000$. So the factorization of a number less than 1 million
has at most nineteen $2$ as a prime factor. Therefor, we can fix the
iteration of computing the number of prime number 2 to be 19. Similarly
we can fix the iteration numbers for all other prime numbers less than 1,000.
Assume we have an array {\tt Prime\_Nums[168]} that contains all the prime
numbers less than 1,000 and an array {\tt Prime\_Num\_Expo[168]} that
contains the maximum exponent for each prime number, the following
pseudo code shows the core part of our solution.

\begin{verbatim}
Integer Counter(32, 0, PUBLIC);
for (int i = 0; i < 168; i++) {
    Integer Prime(32, Prime_Nums[i], Public);
    for (int j = 0; j < Prime_Num_Expo[i]; j++) {
        if ((Input.equal(ZERO.reveal<bool>()) {
            Counter = Counter + ONE;
            Input = Input / Prime;
        }
    }
}
\end{verbatim}
